%% This is an example first chapter.  You should put chapter/appendix that you
%% write into a separate file, and add a line \include{yourfilename} to
%% main.tex, where `yourfilename.tex' is the name of the chapter/appendix file.
%% You can process specific files by typing their names in at the 
%% \files=
%% prompt when you run the file main.tex through LaTeX.

\chapter{Introducción}

El contenido textual es una parte fundamental para la publicidad en internet. Una buena elección de palabras es indispensable para la creación de un texto publicitario, de esto depende el sentido que se da al texto y afecta positiva o negativamente en el interés de los consumidores. Este trabajo presenta un método para la optimización de textos publicitarios mediante técnicas de computación evolutiva interactiva. 

El texto publicitario se divide en bloques de texto que al combinarlos se obtiene una gran cantidad de versiones de ese anuncio.

Los textos publicitarios pueden ser de longitud arbitraria, pero para los experimentos llevados a cabo se utilizan textos cortos. Para evolucionar un texto publicitario, primero se tiene que seguir un un formato propuesto por nosotros, descrito en la Sección 5.1. Este formato permite la representación de cada versión del texto en un cromosoma que puede ser utilizado por un algoritmo genético (ver Sección 3.3.1) para su evolución. 

Las distintas versiones del anuncio fueron presentadas a usuarios para que ellos eligieran la versión que más les atraía, las versiones fueron evolucionando en una plataforma llamada EvoSpace para obtener mediante la evolución una combinación que fuese más atractiva para la mayoría de los usuarios. 

\clearpage
\section{Descripción de la investigación}

Cuando se trata de comercio electrónico el texto toma un papel muy importante porque a través de el llevas la información del articulo en venta al consumidor. \cite{choi2002antecedents} El texto también puede ayudar a convencer al lector de comprar el producto que se anuncia. Cuando la persona encargada de el anuncio publicitario es un experto en el área, el producto obtiene mejores respuestas de los consumidores. La mezcla de las palabras o frases (bloques de texto) que los expertos deciden utilizar al escribir el texto es importante, porque esta mezcla puede ser la decisiva que provoque que el consumidor adquiera ese producto. Si un no experto desea escribir un anuncio publicitario seria muy difícil para él elegir la combinación correcta de los bloques de texto que serán de agrado para la mayoría de los consumidores, lo más factible seria escribir un texto que encierre el mensaje que quiere transmitir y dárselo a un experto para que este lo optimice segun su experiencia y sus conocimientos. 


Los algoritmos evolutivos son comúnmente usados para resolver problemas de optimización \cite{jong2006evolutionary} y es por eso que recurrimos a este método cuando nos interesamos en optimizar anuncios publicitarios. Creemos que si un grupo de personas puede evaluar distintas combinaciones del mismo anuncio, a través de varias generaciones se puede encontrar el anuncio optimo que sea del agrado de la mayoría de los consumidores.


\clearpage
\section{Objetivo General de la Investigación}

Se propone en este trabajo adaptar la plataforma EvoSpace-Interactivo para que optimice textos publicitarios, reescribiendo ciertos bloques de texto, con el fin de conseguir un texto que persuada más a comprar el producto, sea recordado y reconocido con mayor facilidad para la mayoría de  los usuarios.

Proponemos conseguir esta optimización a través de técnicas de computación evolutiva interactiva.

\clearpage
\section{Específicos de la Investigación}

Los objetivos específicos a desarrollar en este trabajo de investigación son los siguientes:

\begin{enumerate}
\item Adaptar la plataforma EvoSpace-Interactivo para optimizar textos publicitarios mediante técnicas de computación evolutiva interactiva.
\item Adaptar EvoSpace-Interactivo para la optimización de los textos publicitarios.
\item Comprobar que se puede optimizar el texto a través de técnicas de computación evolutiva interactiva para una población
\item Validar los resultados de los textos generados con nuestro método.
\item Analizar los resultados de los experimentos realizados.
\end{enumerate}

\clearpage
\section{Estructura del Documento de Tesis}

A continuación se explica brevemente el contenido de los siguientes capítulos que contiene esta tesis:

\begin{description}

\item[Capítulo 2] Se describe detalladamente el problema, las soluciones que existen actualmente y la solución propuesta.

\item[Capítulo 3] Se presenta algunos conceptos básicos para entender esta investigación y también partes esenciales que componen este proyecto como la plataforma utilizada que es una herramienta crucial para la implementación del algoritmo genético utilizado en esta investigación.

\item[Capítulo 4] Se incluyen los antecedentes de métodos utilizados para optimización de los distintos tipos de publicidad en internet,  ya que estos métodos son el fundamento para el desarrollo de este método propuesto.

\item[Capítulo 5] Se propone una arquitectura general para la optimización de textos, se describen y justifican los principales componentes y sus interacciones.

\item[Capítulo 6] Se detalla la implementación y adaptación de la plataforma EvoSpace-Interactivo, especificando las modificaciones realizadas. Se inicia el capitulo con una descripción breve del funcionamiento de de EvoSpace-Interactivo y la razón por la que es ideal para nuestra investigación.

\item[Capítulo 7] Se muestran los distintos experimentos realizado y sus  resultados.

\item[Capítulo 8] Se discuten los resultados y conclusiones sobre el trabajo expuesto, además se presentan propuestas para trabajos futuros.

\end{description}

