% -*-latex-*-
% 
% For questions, comments, concerns or complaints:
% thesis@mit.edu
% 
%
% $Log: cover.tex,v $
% Revision 1.8  2008/05/13 15:02:15  jdreed
% Degree month is June, not May.  Added note about prevdegrees.
% Arthur Smith's title updated
%
% Revision 1.7  2001/02/08 18:53:16  boojum
% changed some \newpages to \cleardoublepages
%
% Revision 1.6  1999/10/21 14:49:31  boojum
% changed comment referring to documentstyle
%
% Revision 1.5  1999/10/21 14:39:04  boojum
% *** empty log message ***
%
% Revision 1.4  1997/04/18  17:54:10  othomas
% added page numbers on abstract and cover, and made 1 abstract
% page the default rather than 2.  (anne hunter tells me this
% is the new institute standard.)
%
% Revision 1.4  1997/04/18  17:54:10  othomas
% added page numbers on abstract and cover, and made 1 abstract
% page the default rather than 2.  (anne hunter tells me this
% is the new institute standard.)
%
% Revision 1.3  93/05/17  17:06:29  starflt
% Added acknowledgements section (suggested by tompalka)
% 
% Revision 1.2  92/04/22  13:13:13  epeisach
% Fixes for 1991 course 6 requirements
% Phrase "and to grant others the right to do so" has been added to 
% permission clause
% Second copy of abstract is not counted as separate pages so numbering works
% out
% 
% Revision 1.1  92/04/22  13:08:20  epeisach

% NOTE:
% These templates make an effort to conform to the MIT Thesis specifications,
% however the specifications can change.  We recommend that you verify the
% layout of your title page with your thesis advisor and/or the MIT 
% Libraries before printing your final copy.


\begin{titlepage}
  \begin{center}
    \includegraphics[width=0.75\textwidth]{./logos.png}~\\[1cm]
    \textsc{\LARGE Optimización de textos publicitarios utilizando técnicas de computación evolutiva interactiva}~\\[0.5cm]
\textsc{por}~\\[0.5cm]
\textsc{\Large Elizabeth Quetzali Madera Hernández}~\\[2.0cm]

\textsc{\Large División de Estudios de Posgrado e Investigación}~\\[0.5cm]
\textsc{Tesis para obtener el grado de}~\\[0.5cm]
\textsc{\Large Maestra en Ciencias Computacionales}~\\[0.5cm]
\textsc{en el}~\\[0.5cm]
\textsc{\Large INSTITUTO TECNÓLOGICO DE TIJUANA}~\\[1cm]
\textsc{Julio 2014}~\\[0.2cm]
\textsc{Tijuana, Baja California, México}~\\[1cm]

\begin{minipage}{1.5\textwidth}
  \begin{flushleft} \large
    \emph{Director:}\\
    Dr. Jose Mario \textsc{García Valdez}
  \end{flushleft}
\end{minipage}

\begin{minipage}{1.5\textwidth}
\begin{flushleft} \large
\emph{Co-Directora:} \\
M.Cs. Alejandra \textsc{Mancilla Soto}
\end{flushleft}
\end{minipage}

\vfill

  \end{center}

\end{titlepage}

% Make the titlepage based on the above information.  If you need
% something special and can't use the standard form, you can specify
% the exact text of the titlepage yourself.  Put it in a titlepage
% environment and leave blank lines where you want vertical space.
% The spaces will be adjusted to fill the entire page.  The dotted
% lines for the signatures are made with the \signature command.
%\maketitle

% The abstractpage environment sets up everything on the page except
% the text itself.  The title and other header material are put at the
% top of the page, and the supervisors are listed at the bottom.  A
% new page is begun both before and after.  Of course, an abstract may
% be more than one page itself.  If you need more control over the
% format of the page, you can use the abstract environment, which puts
% the word "Abstract" at the beginning and single spaces its text.

%% You can either \input (*not* \include) your abstract file, or you can put
%% the text of the abstract directly between the \begin{abstractpage} and
%% \end{abstractpage} commands.

% First copy: start a new page, and save the page number.

% Uncomment the next line if you do NOT want a page number on your
% abstract and acknowledgments pages.
% \pagestyle{empty}
\setcounter{savepage}{\thepage}
\begin{abstractpage}
% $Log: abstract.tex,v $
% Revision 1.1  93/05/14  14:56:25  starflt
% Initial revision
% 
% Revision 1.1  90/05/04  10:41:01  lwvanels
% Initial revision
% 
%
%% The text of your abstract and nothing else (other than comments) goes here.
%% It will be single-spaced and the rest of the text that is supposed to go on
%% the abstract page will be generated by the abstractpage environment.  This
%% file should be \input (not \include 'd) from cover.tex.

La computación Evolutiva Interactiva (CEI) se utiliza en este trabajo con el fin de llevar a cabo la optimización de varios bloques de texto publicitarios. Los anuncios de texto siguen un formato similar al utilizado en una técnica llamada “Article Spinning”. Este formato permite que el algoritmo CEI evolucione el texto para un determinado grupo de personas, usando palabras y frases como partes variables que cambian de acuerdo a la evaluación subjetiva de la gente que interactúa con el algoritmo. Después de varias generaciones, el algoritmo de CEI da como resultado una versión del texto publicitario que, en teoría, debería exhibir un incremento en rendimiento, de acuerdo a la función de evaluación subjetiva con la cual fue evolucionado. Para poder demostrar la eficiencia de los textos, éstos son comparados contra una versión determinada por un experto en algún campo relacionado a la mercadotecnia. Para esta comparación, tres pruebas fueron realizadas: pruebas de memoria, reconocimiento, y persuasión. Los resultados obtenidos muestran que la CEI puede ser usada efectivamente para incrementar el impacto de un texto publicitario, pero más experimentos necesitan ser realizados.

\section*{Abstract}
Interactive Evolutionary Computation (IEC) is used in this work in order to perform the optimization of several advertisement blocks of text. The advertisement texts follow a format similar to the one used in a technique called Article Spinning. This format allows an IEC algorithm to evolve the text for a certain group of people, using words and phrases as variable parts which change according to the subjective evaluation of the people interacting with the algorithm. After several generations, the IEC algorithm provides a version of the advertisement text that, in theory, should exhibit an increased performance, according to the subjective evaluation function it was evolved with. In order to demonstrate the efficiency of the texts, these are compared against a version determined by an expert in a field related to marketing. For this comparison, three tests are performed: recall, recognition, and persuasion tests. The results obtained show that IEC could effectively be used to increase the impact of an advertisement text, but more experiments need to be conducted.

\end{abstractpage}

% Additional copy: start a new page, and reset the page number.  This way,
% the second copy of the abstract is not counted as separate pages.
% Uncomment the next 6 lines if you need two copies of the abstract
% page.
% \setcounter{page}{\thesavepage}
% \begin{abstractpage}
% % $Log: abstract.tex,v $
% Revision 1.1  93/05/14  14:56:25  starflt
% Initial revision
% 
% Revision 1.1  90/05/04  10:41:01  lwvanels
% Initial revision
% 
%
%% The text of your abstract and nothing else (other than comments) goes here.
%% It will be single-spaced and the rest of the text that is supposed to go on
%% the abstract page will be generated by the abstractpage environment.  This
%% file should be \input (not \include 'd) from cover.tex.

La computación Evolutiva Interactiva (CEI) se utiliza en este trabajo con el fin de llevar a cabo la optimización de varios bloques de texto publicitarios. Los anuncios de texto siguen un formato similar al utilizado en una técnica llamada “Article Spinning”. Este formato permite que el algoritmo CEI evolucione el texto para un determinado grupo de personas, usando palabras y frases como partes variables que cambian de acuerdo a la evaluación subjetiva de la gente que interactúa con el algoritmo. Después de varias generaciones, el algoritmo de CEI da como resultado una versión del texto publicitario que, en teoría, debería exhibir un incremento en rendimiento, de acuerdo a la función de evaluación subjetiva con la cual fue evolucionado. Para poder demostrar la eficiencia de los textos, éstos son comparados contra una versión determinada por un experto en algún campo relacionado a la mercadotecnia. Para esta comparación, tres pruebas fueron realizadas: pruebas de memoria, reconocimiento, y persuasión. Los resultados obtenidos muestran que la CEI puede ser usada efectivamente para incrementar el impacto de un texto publicitario, pero más experimentos necesitan ser realizados.

\section*{Abstract}
Interactive Evolutionary Computation (IEC) is used in this work in order to perform the optimization of several advertisement blocks of text. The advertisement texts follow a format similar to the one used in a technique called Article Spinning. This format allows an IEC algorithm to evolve the text for a certain group of people, using words and phrases as variable parts which change according to the subjective evaluation of the people interacting with the algorithm. After several generations, the IEC algorithm provides a version of the advertisement text that, in theory, should exhibit an increased performance, according to the subjective evaluation function it was evolved with. In order to demonstrate the efficiency of the texts, these are compared against a version determined by an expert in a field related to marketing. For this comparison, three tests are performed: recall, recognition, and persuasion tests. The results obtained show that IEC could effectively be used to increase the impact of an advertisement text, but more experiments need to be conducted.

% \end{abstractpage}

\cleardoublepage

\section*{Agradecimientos}

Quiero agradecer principalmente a mis cuatro mejores amigos:

Mi papá por todos sus sabios consejos que nunca me dejan desviar mi camino, por apoyarme en cualquier decisión que he tomado y por ser siempre mi héroe.

Mi mamá por demostrarme que una mujer puede acercarse mucho a la perfección si se lo propone y por enseñarme a ser autosuficiente.

Mi abuela por estar siempre disponible cuando la necesito y por todas esas platicas tan interesantes que me ha dado.

Mi novio Amaury por estar siempre a mi lado, por hacerme la mujer mas feliz del mundo y por ser mi profesor en muchas ocasiones.

Por haberme educado, haber creado una buena atmósfera donde pudiese crecer y por haber alimentado mi mente con sabiduría y amor, cualquier logro mío es también un logro de ustedes. Los amo con toda mi mente y mi corazón.

También quiero agradecer a todos mis profesores porque mi personalidad y mi forma de pensar están construidas con los pedacitos de conocimiento que me han brindado.

Muchas gracias Dr. Mario Garcia por su paciencia, conocimiento y orientación en cada paso que di durante mi investigacion, supo guiarme en todo momento.

Gracias M.Cs. Alejandra Mancilla por las sugerencias y aportaciones que me brindo cuando acudí a usted, fue de gran ayuda tenerla cerca porque aportaba ideas desde un punto de vista distinto al mío y al del Dr. Mario. Gracias a los dos porque fueron para mi un gran equipo.

Dra. Patricia Melin quiero agradecerle por inspirarme, conocer a una mujer que a llegado tan alto me da fuerzas para seguir estudiando. Usted es un gran ejemplo para mi.

Dr. Oscar Castillo gracias por su apoyo, su amabilidad y por sus consejos en cada seminario que fueron de gran ayuda y estimulo.

Gracias Dr. Jose Soria por sus correcciones y comentarios porque gracias a ellos pude entender con mayor profundidad la complejidad de algunos problemas que después pude resolver con mayor facilidad.

Me siento afortuna por haber tenido a tan buenos mentores. Es un orgullo decir que mi educación de posgrado fue impartida por ustedes.

Gracias desde el fondo de mi corazón a cada uno de ustedes. siempre ocuparan un lugar especial dentro de mi.


%%%%%%%%%%%%%%%%%%%%%%%%%%%%%%%%%%%%%%%%%%%%%%%%%%%%%%%%%%%%%%%%%%%%%%
% -*-latex-*-
