%% This is an example first chapter.  You should put chapter/appendix that you
%% write into a separate file, and add a line \include{yourfilename} to
%% main.tex, where `yourfilename.tex' is the name of the chapter/appendix file.
%% You can process specific files by typing their names in at the 
%% \files=
%% prompt when you run the file main.tex through LaTeX.

\chapter{Descripción del problema}

El contenido textual es una parte fundamental para la publicidad en internet. Una buena elección de palabras es indispensable para la creación de un texto publicitario, de esto depende el sentido que se da al texto y afecta positiva o negativamente en el interés de los consumidores. El 97\% de los usuarios al encontrarse con sitios web nuevo escanean el texto que aparece para reconocer palabras y frases individuales con el fin de encontrar algo de interés.\cite{nielsen1997users} Como resultado, las páginas Web tienen que emplear texto susceptible de ser analizada utilizando conjuntos de palabras claves significativas que atraigan a los usuario. 

Contratar a un experto para crear contenido que pueda agradarle a la mayoría de los usuarios que entren a un sitio web puede ser una solución muy costosa, aun asi existe una creciente demanda de los escritores de contenido web especializados en Internet. Esto se debe a la calidad del contenido a menudo se traduce en mayores ingresos para los negocios en línea.\cite{kulkarni2014choose}

\section{Soluciones Existentes Actualmente}

Un escritor de contenido web es una persona que se especializa en la creación de contenido relevante para los sitios web. Cada sitio web tiene un publico objetivo especifico y requiere un un tipo de contenido. El contenido debe tener palabras clave que atraigan y retengan a los usuarios del sitio web. El contenido debe enfocarse en un tema especifico. El contenido generado debe ser considerado fácil de leer, informativo y agradable de leer. Normalmente se contrata a los escritores de contenido web para el desarrollo de contenido de sitios enfocados en la venta de artículos. En la tabla \ref{precios} se pueden observar los precios por palabra de las 10 empresas más populares dedicadas a la creación de contenido.

\begin{table}
\caption{Precios de compañías especializadas en creación de contenido web}
\label{precios}
\begin{center}
\begin{tabular}{|r|c|l|}
Compañia & Precio & Clasificación \\\hline \hline

Media Shower	 & \$49 dlls por publicación & ****\\
WriterAccess & \$50 dlls de depósito; de \$0.02 a \$1 por palabra & ****\\
CopyPress	 & \$5 dlls por 100 palabras & ****\\
Zerys	 & Los precios varian segun el proyecto & ***\\
Articlez & Los precios varian segun el proyecto & ***\\
SEO Article Writing Pros & Los precios varian segun el proyecto & ***\\
iWriter	 & Los precios varian segun el proyecto & ***\\
Constant Content	 & de \$20 a \$250 dlls dependiendo de la calidad & ***\\
Textbroker & de \$1.20 a \$6.70 cada 100 palabras & **\\
oDesk	 & Los precios varian segun el proyecto & *\\

\end{tabular}
\end{center}
\end{table}


\subsection{Solución Propuesta}

Este trabajo se presenta un método para la optimización de textos publicitarios mediante técnicas de computación evolutiva interactiva que pretende ser una solución más factible y alcanzable para la mayoría de interesados en el comercio electrónico.

El texto publicitario se divide en bloques de texto que al combinarlos se obtiene una gran cantidad de versiones de ese anuncio. Cada versión es representada por un vector que tiene la función de un cromosoma. Las distintas versiones del anuncio fuerón presentadas a usuarios para que ellos eligieran la versión que más les atraía, las versiones fueron evolucionando en una plataforma llamada EvoSpace para obtener mediante la evolución una combinación que fuese más atractiva para la mayoría de los usuarios. 

\section{Justificacion}

La razón para el desarrollo de la metodología propuesta en este trabajo es el potencial de reducir los costos monetarios en campañas de publicidad, el texto juega un papel muy importante en el comercio electrónico, ya que este es una de las formas más comunes de dar información acerca de un producto comercial a los consumidores \cite{mcquarrie1999visual}. Al desarrollar este método que optimiza textos publicitarios pretendemos principalmente alcanzar estas 3 metas:

\begin{enumerate}
\item Incrementar las ganancias del comercio electrónico personalizando las descripciones del  los productos para una población. 
\item Crear un nuevo método de optimización de anuncios.
\item Disminuir costos en campañas publicitarias
\end{enumerate}