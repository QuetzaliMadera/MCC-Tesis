%% This is an example first chapter.  You should put chapter/appendix that you
%% write into a separate file, and add a line \include{yourfilename} to
%% main.tex, where `yourfilename.tex' is the name of the chapter/appendix file.
%% You can process specific files by typing their names in at the 
%% \files=
%% prompt when you run the file main.tex through LaTeX.

\chapter{Antecedentes}

En 2007, Kazienko y Adamski propusieron la extracción de patrones de usuario a través del uso de contenido web y técnicas de minería de uso web, y la creación de clusters en base a esta información \cite{kazienko2007adrosa}. Zheng, Chen, y Jiang, 2012, compararon el rendimiento de anuncios con sólo texto, sólo imágenes, y texto e imágenes, y concluyeron que no hay diferencia significativa entre sus rendimientos \cite{zheng2012ontology}. Keng y Liu, 2013, analizaron cómo es que los sitios web necesitan ser diseñados de acuerdo a la personalidad de los usuarios y sus intereses. \cite{keng2013can}. Wu, Zongda, et. al., 2013, se enfocaron en el posicionamiento óptimo de la publicidad, en lugar de la selección del anuncio para ser mostrado al usuario \cite{wu2013position}. Fan, Teng-Kai, y Chia-Hui Chang, 2011, propusieron un marco de trabajo de software para analizar el contenido de blogs, determinar el tema de su contenido, y recomendar anuncioos que fueran relevantes de acuerdo al contenido del blog \cite{fan2011blogger}. 

Dao, Tuan Hung, Seung Ryul Jeong, y Hyunchul Ahn, 2012, desarrollaron una herramiento que recomienda anuncios de acuerdo al contexto de un individuo. Usan la locación del usuario para proveer mejores anuncios. Su sistema de recomendación está basado en filtrado colaborativo \cite{dao2012novel}. Hsieh, Yu-Chen, y Kuo-Hsiang Chen, 2011, y Lewis, Whitler y Hoeg, 2013, estudiaron cómo el tipo de contenido (videos, texto, imágenes, o una mezcla de estos tipos) de un sitio web afectan la atención de un usuario hacia los anuncios de un sitio web \cite{hsieh2011different} \cite{lewis2013customer}. Finalmente, la prueba de persuación usada en este trabajo ha sido utilizada anteriormente en \cite{madera2014ad}.




