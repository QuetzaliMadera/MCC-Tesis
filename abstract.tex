% $Log: abstract.tex,v $
% Revision 1.1  93/05/14  14:56:25  starflt
% Initial revision
% 
% Revision 1.1  90/05/04  10:41:01  lwvanels
% Initial revision
% 
%
%% The text of your abstract and nothing else (other than comments) goes here.
%% It will be single-spaced and the rest of the text that is supposed to go on
%% the abstract page will be generated by the abstractpage environment.  This
%% file should be \input (not \include 'd) from cover.tex.

La computación Evolutiva Interactiva (CEI) se utiliza en este trabajo con el fin de llevar a cabo la optimización de varios bloques de texto publicitarios. Los anuncios de texto siguen un formato similar al utilizado en una técnica llamada “Article Spinning”. Este formato permite que el algoritmo CEI evolucione el texto para un determinado grupo de personas, usando palabras y frases como partes variables que cambian de acuerdo a la evaluación subjetiva de la gente que interactúa con el algoritmo. Después de varias generaciones, el algoritmo de CEI da como resultado una versión del texto publicitario que, en teoría, debería exhibir un incremento en rendimiento, de acuerdo a la función de evaluación subjetiva con la cual fue evolucionado. Para poder demostrar la eficiencia de los textos, éstos son comparados contra una versión determinada por un experto en algún campo relacionado a la mercadotecnia. Para esta comparación, tres pruebas fueron realizadas: pruebas de memoria, reconocimiento, y persuasión. Los resultados obtenidos muestran que la CEI puede ser usada efectivamente para incrementar el impacto de un texto publicitario, pero más experimentos necesitan ser realizados.

\section*{Abstract}
Interactive Evolutionary Computation (IEC) is used in this work in order to perform the optimization of several advertisement blocks of text. The advertisement texts follow a format similar to the one used in a technique called Article Spinning. This format allows an IEC algorithm to evolve the text for a certain group of people, using words and phrases as variable parts which change according to the subjective evaluation of the people interacting with the algorithm. After several generations, the IEC algorithm provides a version of the advertisement text that, in theory, should exhibit an increased performance, according to the subjective evaluation function it was evolved with. In order to demonstrate the efficiency of the texts, these are compared against a version determined by an expert in a field related to marketing. For this comparison, three tests are performed: recall, recognition, and persuasion tests. The results obtained show that IEC could effectively be used to increase the impact of an advertisement text, but more experiments need to be conducted.
